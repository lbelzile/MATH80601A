% Options for packages loaded elsewhere
\PassOptionsToPackage{unicode}{hyperref}
\PassOptionsToPackage{hyphens}{url}
%
\documentclass[
  11pt,
  letterpaper,
]{scrbook}

\usepackage{amsmath,amssymb}
\usepackage{iftex}
\ifPDFTeX
  \usepackage[T1]{fontenc}
  \usepackage[utf8]{inputenc}
  \usepackage{textcomp} % provide euro and other symbols
\else % if luatex or xetex
  \usepackage{unicode-math}
  \defaultfontfeatures{Scale=MatchLowercase}
  \defaultfontfeatures[\rmfamily]{Ligatures=TeX,Scale=1}
\fi
\usepackage{lmodern}
\ifPDFTeX\else  
    % xetex/luatex font selection
\fi
% Use upquote if available, for straight quotes in verbatim environments
\IfFileExists{upquote.sty}{\usepackage{upquote}}{}
\IfFileExists{microtype.sty}{% use microtype if available
  \usepackage[]{microtype}
  \UseMicrotypeSet[protrusion]{basicmath} % disable protrusion for tt fonts
}{}
\makeatletter
\@ifundefined{KOMAClassName}{% if non-KOMA class
  \IfFileExists{parskip.sty}{%
    \usepackage{parskip}
  }{% else
    \setlength{\parindent}{0pt}
    \setlength{\parskip}{6pt plus 2pt minus 1pt}}
}{% if KOMA class
  \KOMAoptions{parskip=half}}
\makeatother
\usepackage{xcolor}
\setlength{\emergencystretch}{3em} % prevent overfull lines
\setcounter{secnumdepth}{5}
% Make \paragraph and \subparagraph free-standing
\ifx\paragraph\undefined\else
  \let\oldparagraph\paragraph
  \renewcommand{\paragraph}[1]{\oldparagraph{#1}\mbox{}}
\fi
\ifx\subparagraph\undefined\else
  \let\oldsubparagraph\subparagraph
  \renewcommand{\subparagraph}[1]{\oldsubparagraph{#1}\mbox{}}
\fi


\providecommand{\tightlist}{%
  \setlength{\itemsep}{0pt}\setlength{\parskip}{0pt}}\usepackage{longtable,booktabs,array}
\usepackage{calc} % for calculating minipage widths
% Correct order of tables after \paragraph or \subparagraph
\usepackage{etoolbox}
\makeatletter
\patchcmd\longtable{\par}{\if@noskipsec\mbox{}\fi\par}{}{}
\makeatother
% Allow footnotes in longtable head/foot
\IfFileExists{footnotehyper.sty}{\usepackage{footnotehyper}}{\usepackage{footnote}}
\makesavenoteenv{longtable}
\usepackage{graphicx}
\makeatletter
\def\maxwidth{\ifdim\Gin@nat@width>\linewidth\linewidth\else\Gin@nat@width\fi}
\def\maxheight{\ifdim\Gin@nat@height>\textheight\textheight\else\Gin@nat@height\fi}
\makeatother
% Scale images if necessary, so that they will not overflow the page
% margins by default, and it is still possible to overwrite the defaults
% using explicit options in \includegraphics[width, height, ...]{}
\setkeys{Gin}{width=\maxwidth,height=\maxheight,keepaspectratio}
% Set default figure placement to htbp
\makeatletter
\def\fps@figure{htbp}
\makeatother
\newlength{\cslhangindent}
\setlength{\cslhangindent}{1.5em}
\newlength{\csllabelwidth}
\setlength{\csllabelwidth}{3em}
\newlength{\cslentryspacingunit} % times entry-spacing
\setlength{\cslentryspacingunit}{\parskip}
\newenvironment{CSLReferences}[2] % #1 hanging-ident, #2 entry spacing
 {% don't indent paragraphs
  \setlength{\parindent}{0pt}
  % turn on hanging indent if param 1 is 1
  \ifodd #1
  \let\oldpar\par
  \def\par{\hangindent=\cslhangindent\oldpar}
  \fi
  % set entry spacing
  \setlength{\parskip}{#2\cslentryspacingunit}
 }%
 {}
\usepackage{calc}
\newcommand{\CSLBlock}[1]{#1\hfill\break}
\newcommand{\CSLLeftMargin}[1]{\parbox[t]{\csllabelwidth}{#1}}
\newcommand{\CSLRightInline}[1]{\parbox[t]{\linewidth - \csllabelwidth}{#1}\break}
\newcommand{\CSLIndent}[1]{\hspace{\cslhangindent}#1}

% \usepackage{amsmath,amssymb,mathtools}
\usepackage{enumerate}
\usepackage{geometry}
\geometry{hmargin=1.2in}

\usepackage{booktabs}
\usepackage{amssymb}
\makeatletter
\def\thm@space@setup{%
  \thm@preskip=8pt plus 2pt minus 4pt
  \thm@postskip=\thm@preskip
}
\makeatother

\usepackage{framed,color}
\definecolor{shadecolor}{RGB}{248,248,248}

\renewcommand{\textfraction}{0.05}
\renewcommand{\topfraction}{0.8}
\renewcommand{\bottomfraction}{0.8}
\renewcommand{\floatpagefraction}{0.75}

%\let\oldhref\href
%\renewcommand{\href}[2]{#2\footnote{\url{#1}}}

\ifxetex
  \usepackage{letltxmacro}
  \setlength{\XeTeXLinkMargin}{1pt}
  \LetLtxMacro\SavedIncludeGraphics\includegraphics
  \def\includegraphics#1#{% #1 catches optional stuff (star/opt. arg.)
    \IncludeGraphicsAux{#1}%
  }%
  \newcommand*{\IncludeGraphicsAux}[2]{%
    \XeTeXLinkBox{%
      \SavedIncludeGraphics#1{#2}%
    }%
  }%
\fi

\makeatletter
\newenvironment{kframe}{%
\medskip{}
\setlength{\fboxsep}{.8em}
 \def\at@end@of@kframe{}%
 \ifinner\ifhmode%
  \def\at@end@of@kframe{\end{minipage}}%
  \begin{minipage}{\columnwidth}%
 \fi\fi%
 \def\FrameCommand##1{\hskip\@totalleftmargin \hskip-\fboxsep
 \colorbox{shadecolor}{##1}\hskip-\fboxsep
     % There is no \\@totalrightmargin, so:
     \hskip-\linewidth \hskip-\@totalleftmargin \hskip\columnwidth}%
 \MakeFramed {\advance\hsize-\width
   \@totalleftmargin\z@ \linewidth\hsize
   \@setminipage}}%
 {\par\unskip\endMakeFramed%
 \at@end@of@kframe}
\makeatother

\makeatletter
\@ifundefined{Shaded}{
}{\renewenvironment{Shaded}{\begin{kframe}}{\end{kframe}}}
\makeatother

\newenvironment{rmdblock}[1]
  {
  \begin{itemize}
  \renewcommand{\labelitemi}{
    \raisebox{-.7\height}[0pt][0pt]{
      {\setkeys{Gin}{width=3em,keepaspectratio}\includegraphics{images/#1}}
    }
  }
  \setlength{\fboxsep}{1em}
  \begin{kframe}
  \item
  }
  {
  \end{kframe}
  \end{itemize}
  }
\newenvironment{rmdnote}
  {\begin{rmdblock}{note}}
  {\end{rmdblock}}
\newenvironment{rmdcaution}
  {\begin{rmdblock}{caution}}
  {\end{rmdblock}}
\newenvironment{rmdimportant}
  {\begin{rmdblock}{important}}
  {\end{rmdblock}}
\newenvironment{rmdtip}
  {\begin{rmdblock}{tip}}
  {\end{rmdblock}}
\newenvironment{rmdwarning}
  {\begin{rmdblock}{warning}}
  {\end{rmdblock}}
\usepackage{mathrsfs}
\DeclareMathAlphabet{\mathcrl}{U}{rsfs}{m}{n}
\usepackage{utopia}
\DeclareMathAlphabet{\mathcal}{OMS}{cmsy}{m}{n}
\makeatletter
\makeatother
\makeatletter
\@ifpackageloaded{bookmark}{}{\usepackage{bookmark}}
\makeatother
\makeatletter
\@ifpackageloaded{caption}{}{\usepackage{caption}}
\AtBeginDocument{%
\ifdefined\contentsname
  \renewcommand*\contentsname{Table of contents}
\else
  \newcommand\contentsname{Table of contents}
\fi
\ifdefined\listfigurename
  \renewcommand*\listfigurename{List of Figures}
\else
  \newcommand\listfigurename{List of Figures}
\fi
\ifdefined\listtablename
  \renewcommand*\listtablename{List of Tables}
\else
  \newcommand\listtablename{List of Tables}
\fi
\ifdefined\figurename
  \renewcommand*\figurename{Figure}
\else
  \newcommand\figurename{Figure}
\fi
\ifdefined\tablename
  \renewcommand*\tablename{Table}
\else
  \newcommand\tablename{Table}
\fi
}
\@ifpackageloaded{float}{}{\usepackage{float}}
\floatstyle{ruled}
\@ifundefined{c@chapter}{\newfloat{codelisting}{h}{lop}}{\newfloat{codelisting}{h}{lop}[chapter]}
\floatname{codelisting}{Listing}
\newcommand*\listoflistings{\listof{codelisting}{List of Listings}}
\usepackage{amsthm}
\theoremstyle{definition}
\newtheorem{example}{Example}[chapter]
\theoremstyle{remark}
\AtBeginDocument{\renewcommand*{\proofname}{Proof}}
\newtheorem*{remark}{Remark}
\newtheorem*{solution}{Solution}
\makeatother
\makeatletter
\@ifpackageloaded{caption}{}{\usepackage{caption}}
\@ifpackageloaded{subcaption}{}{\usepackage{subcaption}}
\makeatother
\makeatletter
\@ifpackageloaded{tcolorbox}{}{\usepackage[skins,breakable]{tcolorbox}}
\makeatother
\makeatletter
\@ifundefined{shadecolor}{\definecolor{shadecolor}{rgb}{.97, .97, .97}}
\makeatother
\makeatletter
\makeatother
\makeatletter
\makeatother
\ifLuaTeX
  \usepackage{selnolig}  % disable illegal ligatures
\fi
\IfFileExists{bookmark.sty}{\usepackage{bookmark}}{\usepackage{hyperref}}
\IfFileExists{xurl.sty}{\usepackage{xurl}}{} % add URL line breaks if available
\urlstyle{same} % disable monospaced font for URLs
\hypersetup{
  pdftitle={Bayesian modelling},
  pdfauthor={Léo Belzile},
  hidelinks,
  pdfcreator={LaTeX via pandoc}}

\title{Bayesian modelling}
\author{Léo Belzile}
\date{}

\begin{document}
\frontmatter
\maketitle
\ifdefined\Shaded\renewenvironment{Shaded}{\begin{tcolorbox}[frame hidden, interior hidden, sharp corners, enhanced, borderline west={3pt}{0pt}{shadecolor}, breakable, boxrule=0pt]}{\end{tcolorbox}}\fi

\renewcommand*\contentsname{Table of contents}
{
\setcounter{tocdepth}{2}
\tableofcontents
}
\mainmatter
\bookmarksetup{startatroot}

\hypertarget{welcome}{%
\chapter*{Welcome}\label{welcome}}
\addcontentsline{toc}{chapter}{Welcome}

\markboth{Welcome}{Welcome}

This book is a web complement to MATH 80601A \emph{Bayesian modelling},
a graduate course offered at HEC Montréal.

These notes are licensed under a
\href{http://creativecommons.org/licenses/by-nc-sa/4.0/}{Creative
Commons Attribution-NonCommercial-ShareAlike 4.0 International License}
and were last compiled on Sunday, July 16 2023.

The objective of the course is to provide a hands on introduction to
Bayesian data analysis. The course will cover the formulation,
evaluation and comparison of Bayesian models through examples and
real-data applications.

\bookmarksetup{startatroot}

\hypertarget{introduction}{%
\chapter{Introduction}\label{introduction}}

\hypertarget{probability-and-frequency}{%
\section{Probability and frequency}\label{probability-and-frequency}}

In classical (frequentist) parametric statistic, we treat observations
\(\boldsymbol{Y}\) as realizations of a distribution whose parameters
\(\boldsymbol{\theta}\) are unknown. The \emph{likelihood principle}
states that all information about parameters is encoded by the
likelihood function, which is optimized numerically or analytically to
find the maximum likelihood estimator. This gives a single value for the
parameter, and large-sample theory shows that the resulting estimator is
asymptotically normal under regularity conditions.

The interpretation of probability in the classical statistic is somewhat
counterintuitive and is understood in terms of long run frequency, which
is why we call this approach frequentist statistic. Think of a fair die:
when we state that values \(\{1, \ldots, 6\}\) are equiprobable, what we
mean is that repeatedly tossing the die should result, in large sample,
in each outcome being realized roughly \(1/6\) of the time (the symmetry
of the object also implies they should equally likely). This
interpretation also carries over to confidence intervals. A
\((1-\alpha)\) confidence interval either contains the true parameter
value or it doesn't, so the probability level is only the long-run
proportion of intervals created by the procedure that should contain the
true fixed value, not the probability that a single interval contains
the true value. This is counterintuitive to most.

In practice, the true value of the parameter \(\boldsymbol{\theta}\)
vector is unknown to the practitionner, thus uncertain: Bayesians would
argue that we should treat the latter as a random quantity rather than a
fixed constant to reflect this lack of knowledge. Since different people
may have different knowledge about these potential values, the prior
knowledge is a form of subjective probabilities, meaning they are
individual specific. For example, if you play cards, one person may have
recorded the previous cards that were played, whereas other may not.
They then assign different probability of certain cards being played.

In Bayesian inference, we consider \(\boldsymbol{\theta}\) as random
variables to reflect our lack of knowledge about potential values taken.
Italian scientist Bruno de Finetti, who is famous for the claim
``Probability does not exist'\,', stated in the preface of Finetti
(\protect\hyperlink{ref-deFinetti:1974}{1974}):

\begin{quote}
Probabilistic reasoning --- always to be understood as subjective ---
merely stems from our being uncertain about something. It makes no
difference whether the uncertainty relates to an unforseeable future, or
to an unnoticed past, or to a past doubtfully reported or forgotten: it
may even relate to something more or less knowable (by means of a
computation, a logical deduction, etc.) but for which we are not willing
or able tho make the effort; and so on {[}\ldots{]} The only relevant
thing is uncertainty --- the extent of our knowledge and ignorance. The
actual fact of whether or not the events considered are in some sense
\emph{determined}, or known by other people, and so on, is of no
consequence.
\end{quote}

On page 3, de Finetti continues
(\protect\hyperlink{ref-deFinetti:1974}{Finetti 1974})

\begin{quote}
only subjective probabilities exist --- i.e., the degree of belief in
the occurrence of an event attributed by a given person at a given
instant and with a given set of information.
\end{quote}

The likelihood
\(\mathcal{L}(\boldsymbol{\theta}; \boldsymbol{y}) \equiv p(\boldsymbol{y} \mid \boldsymbol{\theta})\)
is the starting point for Bayesian inference. However, we adjoin to it a
\textbf{prior} distribution \(p(\boldsymbol{\theta})\) that reflects the
prior knowledge about potential values taken by the \(p\)-dimensional
parameter vector, before observing the data \(\boldsymbol{y}\). We thus
seek \(p(\boldsymbol{\theta} \mid \boldsymbol{y})\): the observations
are random variables but inference is performed conditional on the
observed sample. By Bayes' theorem, the posterior distribution
\(p(\boldsymbol{\Theta} \mid \boldsymbol{Y})\) is

\begin{equation}\protect\hypertarget{eq-posterior}{}{
p(\boldsymbol{\Theta} \mid \boldsymbol{Y}) = \frac{p(\boldsymbol{Y} \mid \boldsymbol{\Theta}) p(\boldsymbol{\Theta})}{\int p(\boldsymbol{Y} \mid \boldsymbol{\theta}) p(\boldsymbol{\theta}) \mathrm{d}\ \boldsymbol{\theta}},
}\label{eq-posterior}\end{equation}

so the posterior \(p(\boldsymbol{\theta} \mid \boldsymbol{y})\) is
proportional, as a function of \(\theta\), to the product of the
likelihood and the prior function. The integral in the denominator,
termed marginal likelihood and denoted
\(p(\boldsymbol{Y}) = \mathsf{E}_{\boldsymbol{\theta}}\{p(\boldsymbol{Y} \mid \boldsymbol{\theta})\}\),
is a normalizing constant that makes the right hand side integrate to
unity.

For the posterior to be \textbf{proper}, we need the product on the
right hand side to be integrable. The denominator of
Equation~\ref{eq-posterior} is a normalizing constant so that the
posterior is a distribution. If \(\boldsymbol{\theta}\) is low
dimensional, numerical integration such as quadrature methods can be
used to compute the latter. To obtain the marginal posterior
\(p(\theta_j \mid \boldsymbol{y}) = \int p(\boldsymbol{\theta} \mid \boldsymbol{y}) \mathrm{d} \boldsymbol{\theta}_{-j}\),
additional integration is needed.

When \(\boldsymbol{\theta}\) is high-dimensional, the marginal
likelihood is untractable. This is one of the main challenges of
Bayesian statistics and the popularity and applicability has grown
drastically with the development and popularity of numerical algorithms
Gelfand and Smith (\protect\hyperlink{ref-Gelfand.Smith:1990}{1990}).
Markov chain Monte Carlo methods circumvent the calculation of the
denominator by drawing approximate samples from the posterior.

\hypertarget{bayesian-updating}{%
\subsection{Bayesian updating}\label{bayesian-updating}}

Subjective probabilities imply that different people with different
prior beliefs would arrive at different conclusions. However, as more
data are gathered, we can use Bayes theorem to update these prior
beliefs and update the posterior. In most instances, the relative weight
of the prior relative to the likelihood becomes negligible: if we
consider independent data \(\boldsymbol{y}_1, \boldsymbol{y}_n\)
observed sequentially, then \begin{align*}
p(\boldsymbol{\theta} \mid \boldsymbol{y}_1, \ldots, \boldsymbol{y}_k) &\stackrel{\boldsymbol{\theta}}{\propto} p(\boldsymbol{y}_k \mid \boldsymbol{\theta}) p(\boldsymbol{\theta} \mid \boldsymbol{y}_1, \ldots, \boldsymbol{y}_{k-1})
\\ & \stackrel{\boldsymbol{\theta}}{\propto} \prod_{i=1}^k p(\boldsymbol{y}_i \mid \boldsymbol{\theta}) p(\boldsymbol{\theta})
\end{align*} If data are exchangeable, the order in which observations
are collected and the order of the belief updating is irrelevant to the
full posterior
\(p(\boldsymbol{\theta} \mid \boldsymbol{y}_1, \ldots \boldsymbol{y}_n)\).

\begin{example}[]\protect\hypertarget{exm-covidrapidtest}{}\label{exm-covidrapidtest}

Back in January 2021, the Quebec government was debating whether or not
to distribute antigen rapid test, with
\href{https://www.cbc.ca/news/canada/montreal/quebec-avoids-relying-on-rapid-covid-19-tests-as-pressure-mounts-to-follow-ontario-s-lead-1.5896738}{strong
reluctance} from authorities given the paucity of available resources
and the poor sensitivity.

A Swiss study analyse the efficiency of rapid antigen tests, comparing
them to repeated polymerase chain reaction (PCR) test output, taken as
benchmark (\protect\hyperlink{ref-Jegerlehner:2021}{Jegerlehner et al.
2021}). The results are presented in Table~\ref{tbl-covid19}

\hypertarget{tbl-covid19}{}
\begin{longtable}[]{@{}lrr@{}}
\caption{\label{tbl-covid19}Confusion matrix of Covid test results for
PCR tests versus rapid antigen tests, from Jegerlehner et al.
(\protect\hyperlink{ref-Jegerlehner:2021}{2021}).}\tabularnewline
\toprule\noalign{}
& PCR \(+\) & PCR \(-\) \\
\midrule\noalign{}
\endfirsthead
\toprule\noalign{}
& PCR \(+\) & PCR \(-\) \\
\midrule\noalign{}
\endhead
\bottomrule\noalign{}
\endlastfoot
rapid \(+\) & 92 & 2 \\
rapid \(-\) & 49 & 1319 \\
total & 141 & 1321 \\
\end{longtable}

Estimated seropositivity at the end of January 2021 according to
projections of the Institute for Health Metrics and Evaluation (IHME) of
8.18M out of 38M inhabitants
(\protect\hyperlink{ref-owidcoronavirus}{Mathieu et al. 2020}), a
prevalence of 21.4\%. Assuming the latter holds uniformly over the
country, what is the probability of having Covid if I get a negative
result to a rapid test?

Let \(\text{rapid} -\) (\(\text{rapid} +\)) denote a negative (positive)
rapid test result and \(\mathrm{C}+\) (\(\mathrm{C}-\)) Covid positivity
(negativity). Bayes' formula gives \begin{align*}
\Pr(\text{C}+ \mid \text{rapid} -) & = \frac{\Pr(\text{rapid} - \mid \text{C}+)\Pr(\text{C}+)}{\Pr(\text{rapid} - \mid \text{C}+)\Pr(\text{C}+) + \Pr(\text{rapid} - \mid \text{C}-)\Pr(\text{C}-)} \\&=
\frac{49/141 \cdot 0.214}{49/141 \cdot 0.214 + 1319/1321 \cdot 0.786} \\&\approx 0.0866
\end{align*} so there is a small, but non-negligeable probability that
the rapid test result is misleading. Jegerlehner et al.
(\protect\hyperlink{ref-Jegerlehner:2021}{2021}) indeed found that the
sensitivity was 65.3\% among symptomatic individuals, but dropped down
to 44\% for asymptomatic cases. This may have fueled government experts
scepticism.

\end{example}

\bookmarksetup{startatroot}

\hypertarget{priors}{%
\chapter{Priors}\label{priors}}

\bookmarksetup{startatroot}

\hypertarget{references}{%
\chapter*{References}\label{references}}
\addcontentsline{toc}{chapter}{References}

\markboth{References}{References}

\hypertarget{refs}{}
\begin{CSLReferences}{1}{0}
\leavevmode\vadjust pre{\hypertarget{ref-deFinetti:1974}{}}%
Finetti, Bruno de. 1974. \emph{Theory of Probability: A Critical
Introductory Treatment}. Vol. 1. New York: Wiley.

\leavevmode\vadjust pre{\hypertarget{ref-Gelfand.Smith:1990}{}}%
Gelfand, Alan E., and Adrian F. M. Smith. 1990. {``Sampling-Based
Approaches to Calculating Marginal Densities.''} \emph{Journal of the
American Statistical Association} 85 (410): 398--409.
\url{https://doi.org/10.1080/01621459.1990.10476213}.

\leavevmode\vadjust pre{\hypertarget{ref-Geman.Geman:1984}{}}%
Geman, Stuart, and Donald Geman. 1984. {``Stochastic Relaxation, {G}ibbs
Distributions, and the {B}ayesian Restoration of Images.''} \emph{IEEE
Transactions on Pattern Analysis and Machine Intelligence} PAMI-6 (6):
721--41. \url{https://doi.org/10.1109/TPAMI.1984.4767596}.

\leavevmode\vadjust pre{\hypertarget{ref-Jegerlehner:2021}{}}%
Jegerlehner, Sabrina, Franziska Suter-Riniker, Philipp Jent, Pascal
Bittel, and Michael Nagler. 2021. {``Diagnostic Accuracy of a
{SARS-CoV-2} Rapid Antigen Test in Real-Life Clinical Settings.''}
\emph{International Journal of Infectious Diseases} 109 (August):
118--22. \url{https://doi.org/10.1016/j.ijid.2021.07.010}.

\leavevmode\vadjust pre{\hypertarget{ref-owidcoronavirus}{}}%
Mathieu, Edouard, Hannah Ritchie, Lucas Rodés-Guirao, Cameron Appel,
Charlie Giattino, Joe Hasell, Bobbie Macdonald, et al. 2020.
{``Coronavirus Pandemic (COVID-19).''} \emph{Our World in Data}.

\end{CSLReferences}


\backmatter

\end{document}
